\documentclass[9pt]{article}

% if you need to pass options to natbib, use, e.g.:
%     \PassOptionsToPackage{numbers, compress}{natbib}
% before loading neurips_2020

% ready for submission
% \usepackage{neurips_2020}

% to compile a preprint version, e.g., for submission to arXiv, add add the
% [preprint] option:
\usepackage[preprint]{neurips_2020}

% to compile a camera-ready version, add the [final] option, e.g.:
%     \usepackage[final]{neurips_2020}

% to avoid loading the natbib package, add option nonatbib:
%     \usepackage[nonatbib]{neurips_2020}

\usepackage[utf8]{inputenc} % allow utf-8 input
\usepackage[T1]{fontenc}    % use 8-bit T1 fonts
\usepackage{hyperref}       % hyperlinks
\usepackage{url}            % simple URL typesetting
\usepackage{booktabs}       % professional-quality tables
\usepackage{amsfonts}       % blackboard math symbols
\usepackage{nicefrac}       % compact symbols for 1/2, etc.
\usepackage{microtype}      % microtypography

\usepackage{todonotes}

\usepackage{bm}
\usepackage{amsmath}
\usepackage{graphicx}

\graphicspath{{./../../plots/}}
\newcommand{\real}{\text{real}}

\title{Bayesian Neural Arithmetic Units -- Leveraging The Complex Plane}
\author{
  Niklas Heim,
  V\'aclav \v Sm\'idl,
  Tom\'a\v s Pevn\'y \\
  Artificial Intelligence Center\\
  Czech Technical University\\
  Prague, CZ 120 00\\
  \texttt{\{niklas.heim, vasek.smidl, tomas.pevny\}@aic.fel.cvut.cz}\\
}

\begin{document}

\maketitle

\begin{abstract}
  Common Neural Networks poorly approximate simple arithmetic operations.  The
  new Neural Arithmetic Units aim to overcome this difficulty, but are limited
  either limited to operate on positive numbers (NALU by
  \citet{trask_neural_2018}), or can only represent simple addition and
  multiplication (NAU \& NMU by \citet{madsen_neural_2020}).
  We present the first Neural Arithmetic Unit that operates on the full domain
  of real numbers and is capable of learning arbitrary power functions.
  For a theoretically grounded sparsification of the units we employ Bayesian
  compression on the model weights via Automatic Relevance Determination.
\end{abstract}


\section{Introduction}%
\label{sec:introduction}


It is known that Neural Networks can approximate functions arbitrarily
well\todo{cite universal approx}.  However, this strength of approximation
often comes at the cost of generalization beyond the training data manifold.
\todo{mention ODEs as use case}

\section{Related Work}%
\label{sec:related_work}

\citet{trask_neural_2018} have demonstrated how severe this problem is even for
the simplest arithmetic operations, such as summing or multiplying two numbers.
In order to increase the power of abstraction of Neural Networks they propose a
\emph{Neural Arithmetic Logic Unit} (NALU) which can help to overcome this
problem \cite{trask_neural_2018} and is capable of learning the arithmetic
addition, subtraction, multiplication, and division
$\{+,-,\times,\div\}$ with stunning extrapolation accuracy.  However,
the NALU comes with the severe limitation not being able to handle negative
inputs due to the logarithm in the multiplication part of the NALU:

\begin{align}
  \label{eq:nalu_add}
  \text{Addition: }       & \bm a = \bm W \bm x
                          & \bm W& = \tanh(\hat{\bm W}) \odot \sigma(\hat{\bm M}) \\
  \label{eq:nalu_mult}
  \text{Multiplication: } & \bm m = \exp \bm W(\log(|\bm x|+\epsilon)) & &\\
  \text{Output: }         & \bm y = \bm a \odot \bm g + \bm m \odot (1-\bm g) 
                          & \bm g& = \sigma(\bm G\bm x)
\end{align}

A multiplication layer that can handle negative inputs was introduced by \citet{madsen_neural_2020}.
The \emph{Neural Multiplication Unit} (NMU) is defined by Eq.~\ref{eq:nmu} and is typically used in
conjunction with the so-called \emph{Neural Addition Unit} (NAU) in Eq.~\ref{eq:nau}.
\begin{align}
  \label{eq:nmu}
  \text{NMU: } &y_j = \prod_i M_{ij} z_{i} + 1 - M_{ij}  &M_{ij}=\min(\max(M_{ij}, 0), 1)\\
  \label{eq:nau}
  \text{NAU: } &\bm y = \bm A \bm x &A_{ij}=\min(\max(A_{ij}, 0), 1)
\end{align}
In both NMU and NAU the weights are clipped to $[0,1]$, and typically regularized
with $\mathcal{R}$:
\begin{align}
  \label{eq:rsparse}
  \mathcal{R} = \sum_{ij} \min(W_{ij}, 1-W_{ij})
\end{align}

The combination of NAU and NMU can thus learn $\{+,-,\times\}$, but no division.

\todo{put this in acknowledgements
all the results in this paper were create with the help of the following Julia
packages: \cite{rackauckas_differentialequationsjl_2017} + (Flux.jl)}



\section{Neural Power Unit}%
\label{sec:neural_power_unit}

Our \emph{Neural Power Unit} (NPU) can learn arbitrary power functions (which
includes division) while still being able to correctly deal with negative
inputs.  The NPU layer is defined by
\begin{align}
  k_i &= \begin{cases}
     0  & x_i \leq 0 \\
    \pi & x_i > 0
  \end{cases} \\
  \bm r &= |\bm x| \\
  \bm m &= \exp(\bm M \log(\bm r)) \odot \cos(\bm M \bm k)
\end{align}
where $\bm k$ is a vector that is zero where $\bm x$ is positive and $\pi$
where it is negative, $\bm r$ is the elementwise absolute value, and $\bm M$
the multiplication matrix that we want to learn.

The NPU is inspired by the multiplication part of the NALU
Eq.~\ref{eq:nalu_mult}, extended to negative inputs by taking advantage of the
complex logarithm. In general, for any complex number $z$\footnote{Note that
complex numbers can be represented in the polar, complex plane where
$z=re^{i\varphi}$ with $r>0$ and $\varphi \in (0,2\pi)$.}
\begin{align}
  \log(z) = \log\left(r\cdot e^{i\varphi}\right)
     = \log(r) + i\varphi.
\end{align}
However, as $z$ is assumed to be real, we can simplify as follows:
\begin{align}
  \log(z) = \log(r) + ik\pi,
\end{align}
where $k \in [0,1]$ is zero for positive inputs and one for negative inputs $z$.

\begin{align}
  \bm m &= \exp(\bm M \log(\bm x)) \\
    &= \exp(\bm M ( \log(\bm r) + i\pi\bm k ))
\end{align}

\begin{align}
  \bm m_{\text{re}} &= \real(\exp(\bm M \log \bm r + i\pi\bm M\bm k)) \\
    &= \real(\exp(\bm M \log \bm r) \odot (\cos(\pi \bm M \bm k) + i \sin(\pi \bm M \bm k))) \\
    &= \exp(\bm M \log \bm r) \odot \cos(\pi \bm M \bm k)
\end{align}

\section{Experimetns}%
\label{sec:experimetns}

\subsection{10 param func}%
\label{sub:10_param_func}


% \begin{figure}
%   \centering
%   \includegraphics[width=0.8\linewidth]{10-param-func-task-nmu-paper.pdf}
%   \caption{MSE optimization result}%
%   \label{fig:10-param-func-task-nmu-paper}
% \end{figure}
% \begin{figure}
%   \centering
%   \includegraphics[width=0.8\linewidth]{10-param-func-bayes-task-nmu-paper.pdf}
%   \caption{ARD result with MSE starting point}%
%   \label{fig:10-param-func-bayes-task-nmu-paper}
% \end{figure}
% \begin{figure}
%   \centering
%   \includegraphics[width=0.8\linewidth]{10-param-func-bayes-task-nmu-paper-history.pdf}
%   \caption{ARD history}%
%   \label{fig:history}
% \end{figure}
% 
% \subsection{10 param func with sqrt and power}%
% \label{sub:10_param_func_with_sqrt_and_power}
% 
% \begin{figure}
%   \centering
%   \includegraphics[width=0.8\linewidth]{10-param-func-task-sqrt-power.pdf}
%   \caption{10-param-func-task-sqrt-Power}%
%   \label{fig:10-param-func-task-sqrt-power}
% \end{figure}
% \begin{figure}
%   \centering
%   \includegraphics[width=0.8\linewidth]{10-param-func-bayes-task-sqrt-power.pdf}
%   \caption{10-param-func-bayes-task-sqrt-Power}%
%   \label{fig:10-param-func-bayes-task-sqrt-power}
% \end{figure}
% \begin{figure}
%   \centering
%   \includegraphics[width=0.8\linewidth]{10-param-func-bayes-task-sqrt-power-history.pdf}
%   \caption{10-param-func-bayes-task-sqrt-Power-history}%
%   \label{fig:10-param-func-bayes-task-sqrt-power-history}
% \end{figure}


\subsection{Gradient surfaces}%
\label{sub:gradient_surfaces}



\bibliographystyle{plainnat}
\bibliography{refs}

\end{document}
